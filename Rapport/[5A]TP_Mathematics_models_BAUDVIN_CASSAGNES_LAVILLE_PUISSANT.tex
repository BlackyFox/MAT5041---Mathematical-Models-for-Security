\documentclass[a4paper,10pt]{article}

\input{conf.tex}

\begin{document}

%\maketitle
\begin{titlepage}
      \begin{center}   
        \Huge
        \textbf{Mathematical Models for Security}
        
        %\vspace{0.5cm}
        \LARGE
        ~
        
%         \vspace{0.5cm}
        
        \vfill
        \begin{figure}[H]
	    \centering
	    \begin{minipage}{0.89\textwidth}
		\centering
		\includegraphics[width=\textwidth]{./img/esiea.jpeg}
	    \end{minipage}
	\end{figure}
        \vfill
        
        \vspace{0.5cm}
        
        Contrôle de zone et théorie des graphes
        
        \vspace{2cm}
        \textbf{Emmanuel BAUDVIN\\Lucien CASSAGNES\\Lucas LAVILLE\\Antoine PUISSANT}\\
        \vspace{0.8cm}
        \Large
        \underline{Enseignant :} M. FILIOL\\
        \vspace{0.5cm}
        2015 - 2016%\today
        
    \end{center}
\end{titlepage}

\begin{abstract}
L'objectif de ce mini-projet est de découvrir une application de la théorie des graphes dans le domaine de la sécurité. Vous allez étudier le problème MCS (Minimum Cut Set ou coupure minimale d'un graphe) dans le contexte du contrôle de zone. Ce dernier peut être envisagé de manière duale :
\begin{itemize}
 \item dans le cas défensif, il s'agit de protéger optimalement une zone donnée et ce, avec des ressources les plus limitées possibles,
 \item dans le cas offensif, par exemple, retarder, dans les mêmes conditions (optimalement et avec des ressources limitées), l'arrivée des forces d'intervention face à une attaque sur une zone donnée (point sensible)
\end{itemize}
\end{abstract}

\newpage

\tableofcontents

\section{Question 1}
\textit{Expliquez ce qu'est le problème du Minimum Cut Set (MCS) en théorie des graphes.}



\section{Question 2}
\textit{Etudiez l'algorithme de Ford-Fulkerson pour résoudre le problème du flot maximal. Décrivez-en le principe et les principales étapes algorithmiques.}\\
Dans cette partie nous allons étudier l'algorihme de Ford-Fulkerson, l'écire en pseudo code puis expliciter chacun des étapes pour en comprendre le fonctionnement.\\
L'algorithme de Ford-Fulkerson (noté FF dans cette partie) est un algorithme qui permet de résoudre le problème du flox maximum dans un flot à une entrée et une sortie. Chacun des ces paramètre est représenté sous la forme de noeud, structure classique des graphes qui sont reliés par des arcs. L'agorithme FF permet d'optimiser les flux dans les réseaux qui sont représentables dans les graphes (d'où le rapport entre le document et son application).\\
Ci-desous le pseudo code de l'algorithme:\\
Les entrées:\begin{itemize}
\item Un graphe G de capacité \textit{c}, noeud de départ \textit{s} et arrivé \textit{t};
\end{itemize}
Les sorties:\begin{itemize}
\item Une flux minimum entre \textit{s} et \textit{t}
\end{itemize}
\textbf{Algorithme \textit{Ford-Fulkerson}}
\begin{enumerate}
\item $f(u,v) \leftarrow 0$ \\ On met le poids des arcs à 0 pour tous les couples $(u,v)$; u et v étant des noeuds appartenant au graphe G.
\item \textbf{Tantque}($p=$ExisteUnChemin(s,t) \& capacité(u,v) $> 0$) \textbf{alors}\\
On itère alors des instructions en s'assurant que la capacité entre les noeuds u et v où l'on se situe permet de faire passer le dit flux dans le graphe. On vérifie aussi que le chemin entre s et t existe toujours.
\begin{itemize}
\item Chercher(capacité\_min(p))\\
On cherche alors la capacité minimale de p, le chemin entre s et t.
\item \textbf{PourChaque} $(u,v)\in p$ \textbf{faire}
\begin{itemize}
\item $f(u,v) \leftarrow f(u,v) + capacite(p)$\\
On envoie le flux à travers le graphe et on met à jour les poids des arcs
\item $f(v,u) \leftarrow f(v,u) - capacite(p)$\\
Cette instruction n'est là que si retourner la valeur du flux minimum à travers le graphe est nécessaire
\end{itemize}
\textbf{fin\_PourChaque}
\end{itemize}
\textbf{fin\_Tantque}
\end{enumerate}

\section{Question 3}
\textit{Expliquez le principe de résolution du problème MCS par l'algorithme de Ford-Fulkerson.}\\~\\\par
On peut résoudre le Minimum Cut Set Problem par l'algorithme de Ford Fulkerson grâce au théorème suivant :\\\par
Si $f$ est un flot dans un graphe $G$, les trois propositions suivantes sont équivalentes :\\
\begin{itemize}
 \item $f$ est un flot maximum (à fortiori, il le sera si on le calcule avec l'algorithme de Ford Fulkerson
 \item Le réseau résiduel du graphe $G$ ne contient pas de chemin améliorant (puisqu'on aura appliqué Ford Fulkerson)
 \item Il existe une coupe du graphe $G$ séparant $s$ et $s$ dans deux sous graphes, dont la capacité vaut $|f|$
\end{itemize}


\section{Question 4}
\textit{Décrivez et expliquez les principaux résultats de la référence et comment les auteurs ont résolus le problème du contrôle de zone via l'algorithme de Ford-Fulkerson}

\newpage
% \rhead{2014-2015}
% \nocite{*}
\printbibliography\addcontentsline{toc}{section}{Références}

\newpage

\appendix
\section{Annexes}
\subsection{Liste des villes étudiées}
\label{annexe:1}
\begin{table}[H]
 \centering
 \begin{tabular}{l@{\hskip 5cm}l}
    San Francisco & Washington DC\\
    San Diego & Boston\\
    Las Vegas & Detroit\\
    Virginia Beach & Phoenix\\
    Jacksonville & Seattle\\
    Salt Lake City & Denver\\
    New Orleans & St. Louis\\
    Miami & Cleveland\\
    Tampa & San Antonio\\
    Baltimore & Cincinnati\\
    Riverside & Milwaukee\\
    Portland & Columbus\\
    Sacramento & Providence\\
    San Jose & Hartford\\
    Orlando & Los Angeles\\
    Austin & Chicago\\
    Charlotte & Philadelphia\\
    Memphis & Dallas\\
    Louisville & Houston\\
    Nashville & Atlanta\\
    Richmond & Minneapolis\\
    Buffalo & Pittsburgh\\
    Bridgeport & Kansas City\\
    Raleigh & Indianapolis\\
    New York City & Oklahoma City\\
 \end{tabular}
\end{table}


% \glsaddall
% \printglossaries

%\listoffigures\addcontentsline{toc}{section}{\listfigurename}
\end{document}
